\chapter{Conclusions}\label{ch:conclusions}

Desprès de revisar els treballs previs i la bibliografia, s'han proposat diferents punts sobre els quals es podia treballar per a millorar el funcionament general d'aquest sistema destinat a la detecció \textit{in situ} de productes tòxics en aigües subterrànies. Per avaluar la viabilitat d'aquestes propostes, s'han analitzat i verificat mitjançant simulacions per ordinador.

La validació d'aquestes propostes s'ha realitzat en un laboratori electrònic que ha permès afinar el comportament del sistema abans d'implementar-lo en una \ac{PCB} prototip i, finalment, fabricar-la. Ha faltat temps per acabar de muntar-la i testejar-la degut a que s'ha hagut de realitzar aquesta memòria .

També s'ha hagut de desenvolupar una eina de \textit{software} que incorpora les funcionalitats proposades que no estaven incloses en anteriors versions.

\section{Futures línies d'investigació i desenvolupament}\label{sec:futures_linies_d'investigacio}

De cara a la continuïtat del projecte, cal acabar de muntar la \ac{PCB} i validar-ne el funcionament abans de poder realitzar proves de detecció. En primer lloc, caldrà analitzar l'eficàcia del sistema amb mostres de laboratori produïdes mitjançant tinys, que serviran per calibrar el sistema de cara a poder realitzar proves amb mostres d'aigua subterrània.

Per augmentar la portabilitat del detector, caldria introduir un mètode de comunicació sense fils que permetés treballar remotament quan calgués utilitzar el dispositiu en zones de difícil accés. Una opció seria desenvolupar un mòdul \textit{Bluetooth} capaç d'enviar el mateix tipus d'ordres provinents del port sèrie de l'ordinador.